\documentclass[]{article}


\usepackage{graphicx}
\usepackage{color}


\renewcommand{\contentsname}{Inhoudsopgave}

\begin{document}
	\pagenumbering{gobble}
\begin{titlepage}
	\centering
	{\bfseries\Huge
	Beleidsplan 2017-2018\\}
	\textit{\\\large Leef, alsof het je laatste dag is}
	\begin{figure}[h]
		\includegraphics[scale=0.6]{beleidsplan1}
	\end{figure}
	\vfill
	\flushleft
	{
		\begin{table}[h]
			\begin{tabular}{ll}
				\textbf{Kandidaatsbestuur 2017-2018}\\
				Voorzitter:                 & Daan de Groot    \\
				Secretaris:                 & Niels van Geel   \\
				Penningmeester:             & Leon Jansen      \\
				Technisch Commissaris:       & Simon van Santen \\
				Commissaris Intern:          & Imre ter Horst   \\
				Commissaris Extern en Promo: & Thomas te Wierik
			\end{tabular}
		\end{table}
	}
\end{titlepage}
\pagenumbering{roman}
\section*{Voorwoord}

\clearpage
\pagenumbering{gobble}
\tableofcontents

\pagenumbering{arabic}
\section{Speerpunten}
\subsection{Technisch Beleid}

\subsection{Activiteitenbeleid}

\subsection{Extern Beleid}

\subsection{Financieel beleid}

\subsection{Promotiebeleid}

\subsection{Digitaal Beleid}

\section{Technisch Beleid}
\subsection{Toss}

\subsection{Training}
\subsubsection{Clubtraining}

\subsubsection{Selectietraining}

\subsubsection{Beginnerstraining}

\subsection{Competitie}
\subsubsection{Voorjaarscompetitie}

\subsubsection{Najaarscompetitie}

\subsubsection{Teambattle}

\subsection{Toernooien}
\subsubsection{Clubkampioenschappen}

\subsubsection{Slow Open}

\section{Activiteiten}
\subsection{InSlowductie}
De InSlowductieperiode is voor Slow de periode om haar nieuwe leden te tonen dat Slow veel meer is dan alleen tennis. Het doel is om de eerstejaars elkaar te laten leren kennen en ook een band tussen eerstejaars en ouderejaars te scheppen. De afgelopen jaren zijn InSlowductieactiviteiten daarom ook veelal opengesteld voor ouderejaars leden. Wij beschouwen deze ontwikkeling als positief en willen daar dus mee doorgaan.\\
Wel willen we al tijdens de open maand al een of twee activiteiten voor de InSlowductie organiseren, zodat nieuwe leden al eerder kennis kunnen maken met het gezelligheidsascpect van Slow. Hierdoor zal het aantal InSlowductieactiviteiten uitkomen op 4 of 5. De activiteitencommissie zal zich ontfermen over de organisatie van deze activiteiten.

\subsection{InSlowductieweekend}
Traditioneel wordt de InSlowductieperiode afgesloten met het InSlowductieweekend. Hierbij gaan de eerstejaarsleden samen naar een buitenlandse stad om elkaar nog beter te leren kennen onder het genot van feestjes, drankjes en een aantal activiteiten. Vorig jaar is voor het eerst gekozen om te werken met een reservelijst waar ouderejaarsleden zich kunnen inschrijven. Wanneer het weekend niet volledig bezet is kunnen deze ouderejaars mee. Afgelopen jaar zijn twee ouderejaarsleden op deze manier meegegaan. Beiden hebben het weekend als leuk ervaren. Wij willen dit daarom graag voortzetten. Het InSlowductieweekend zal georganiseerd worden door de InSlowductieweekendcommissie.

\subsection{Stamkroeg}

\subsection{Feesten}
\subsubsection{Slow in the Dark}

\subsubsection{SHOTS}

\subsubsection{Villafeesten}

\subsubsection{Stamkroegfeesten}

\subsubsection{Gala}

\subsection{Wintersport}

\subsection{Overige Activiteiten}

\section{Externe Betrekkingen}
\subsection{Extern Beleid}

\subsection{SLET}

\subsection{Externe Activiteiten}
\subsubsection{Davis Cup}

\subsubsection{Batavierenrace}

\subsubsection{Activiteiten met andere verenigingen}
\section{Financi\"en}


\section{Vereniging}
\subsection{Leden}

\subsection{Kleding}
De kledinglijn die de afgelopen jaren is aangeboden is goed voor de herkenbaarheid van Slow. We zien nog altijd dat de kleding veel gedragen wordt, dus we zien weinig reden tot verandering. Wel lijkt het ons leuk om ter promotie foto's van (knappe) Slowaken in de kleding te maken. Dit zal worden gedaan door de fotocommissie.

\subsection{Website}

\subsection{Slowmotion}
Met de komst van de nieuwe website is er naar onze mening ruimte om te kijken naar de vorm van ons clubblad. De website biedt de mogelijkheid tot het online publiceren van artikelen dan wel het blad als geheel. Omdat het drukken van het blad een hoop geld kost vinden wij het nodig hierover een discussie op gang te brengen. Enerzijds kost het blad in de huidige vorm veel geld terwijl er een hoop leden zijn die het blad niet eens lezen, anderzijds krijgt de redactiecommissie nog altijd te horen dat de papieren versie zo prettig leest. Daarom bestaan er volgens ons een aantal opties:
\begin{enumerate}
	\item
	De Slowmotion gaat door in de huidige vorm.
	\item
	De Slowmotion wordt voortaan alleen nog online gepubliceerd.
	\item
	De Slowmotion wordt gedrukt voor actieve leden en er zullen een aantal Slowmotions bij de toss neergelegd worden.
	\item
	Leden kunnen op de site aangeven of ze de Slowmotion wensen te ontvangen of niet. Een mogelijkheid is om hieraan kosten te hangen.
\end{enumerate}
Voor al deze opties geldt dat er momenteel een contract ligt met een nieuwe drukker. Mogelijke beslissingen hierover zullen dus pas toepassing zijn na aflopen van het contract. Uiteraard is het online publiceren van artikelen of het gehele blad altijd een optie, ongeacht welke optie de voorkeur geniet.
\subsection{Accommodaties}

\subsection{Algemene Introductie}

\subsection{Bestuursfuncties}
\subsubsection{Voorzitter}

\subsubsection{Secretaris}

\subsubsection{Penningmeester}

\subsubsection{Technisch Commissaris}

\subsubsection{Commissaris Intern}

\subsubsection{Commissaris Extern en Promo}

\subsection{Commissies}
\subsubsection{Bestuursadviescommissie (BAC)}

\subsubsection{Kascontrolecommissie}

\subsubsection{Technische Commissie}

\subsubsection{Activiteitencommissie}

\subsubsection{Tosscommissie}

\subsubsection{Kookcommissie (Nascie)}

\subsubsection{Toernooicommissie}

\subsubsection{Feestcommissie}

\subsubsection{Inslowductieweekend}

\subsubsection{SLET}

\subsubsection{Redactiecommissie}

\subsubsection{Promo -en Designcommissie}
De afgelopen jaren is er een Designcommissie opgezet, maar naar ons idee is deze nooit echt van de grond gekomen. De afgelopen jaren is het ontwikkelen van promotiemateriaal vooral gedaan door Sander van Doorn, en met het minder actief worden van Sander is ook een groot deel van de knowhow verdwenen.\\\\
Afgelopen jaar is met Thomas een nieuw lid gekomen die veel van de promotiematerialen heeft ontworpen. We willen er graag voor zorgen dat deze knowhow in de toekomst beter behouden blijft voor Slow, dus willen we meer aandacht besteden aan deze commissie. Verder willen we hun takenpakket wat uitbreiden. Naast het ontwerpen van promotiemateriaal zullen ze zich ook bezighouden met promotie in algemenere zin: bijvoorbeeld het schrijven van promoteksten. Hiermee wordt de Promocommissie dus betrokken in het promoten van open feesten, SLET en de Slow Open. Deze commissie zal geleid worden door de Commissaris Promo.

\subsubsection{Fotocommissie}
Afgelopen jaar is er een camera gekocht om tijdens activiteiten, toernooien en ter promotie van bijvoorbeeld SET foto's te kunnen maken. Samen met de action-cam biedt dit natuurlijk een hoop mogelijkheden. Bij andere verenigingen zien we regelmatig het bestaan van een fotocommissie, een commissie die foto's maakt en bewerkt. Dit leidt op termijn tot meer en vooral ook betere foto's. Daarom lijkt het ons een goed idee om hiermee te beginnen.\\\\
In deze commissie zitten leden die het leuk vinden om met fotografie bezig te zijn. Hun taak is tijdens activiteiten, toernooien en wellicht ook bij de promotour van SLET foto's te maken.

\subsubsection{Mediacommissie}
Met de komst van de nieuwe website willen we deze natuurlijk meer gaan gebruiken en ook beter onderhouden. Op die manier maken we de investering waar en benutten deze ook beter voor de langere termijn. Het onderhoud van de site zal nog steeds voor het grootste deel door het bestuur gedaan worden.\\\\
Het doel is om frequenter nieuwe content op de site te zetten. Hiertoe willen we een overkoepelende commissie, de Mediacommissie. Deze commissie omvat de Promo -en Designcommissie, de Fotocommissie en (een deel van) de Redactiecommissie. Door af en toe gezamelijk te vergaderen kunnen we de content voor de site volgens een duidelijk plan opzetten.

\subsubsection{Wintersportcommissie}

\subsubsection{Introductiecommissie}

\subsubsection{SUB-weekend}

\subsubsection{Slow-Apeliotes Commissie}

\subsubsection{Batavierenrace}

\subsubsection{Galacommissie}

\subsubsection{Villafeest}

\subsubsection{Sponsorcommissie}
De afgelopen jaren is het steeds lastiger gebleken om sponsoren voor Slow te vinden. Bedrijven willen veel gerichter sponsoren en vaak past een studententennisvereniging niet in hun plaatje. Het is lastig hier verandering in te brengen, omdat wij niet denken dat Sponsorcommissies van de afgelopen jaren dit op een verkeerde manier hebben aangepakt.\\\\
Toch willen we proberen om een aantal nieuwe sponsoren binnen te halen. We kiezen hierbij voor een iets andere opzet, omdat we denken dat geldelijke sponsoren binnenhalen geen realistisch doel is, al blijft dat natuurlijk het ideaalbeeld. We zullen dus kijken om bijvoorbeeld korting bij bepaalde producten te regelen, zoals bijvoorbeeld nu bij Toonen het geval is.\\\\
Tot slot is het doel om meer geld op te halen met Sponsorkliks. Afgelopen jaar is gewerkt aan een applicatie voor op je computer. Deze geeft je een melding als je een webwinkel bezoekt die gelinkt is aan Sponsorkliks, zodat het makkelijker wordt om Sponsorkliks te gebruiken. Door het gebruik van deze applicatie actief te promoten hopen we hier meer geld mee op te halen.

\end{document}          
